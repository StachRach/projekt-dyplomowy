\chapter*{Abstract}
\indent The project focuses on aspects of automating the diagnostic performance of a laser-based tomograph. After a description of the LCT, the physical aspects of accelerometers and gyroscopes on which the prototype sensor with controller is based are presented. The Kalman filter is also addressed, and why its use is needed. Next, the premise and operation of the Arduino platform and the SEN0142 module are introduced. The next section presents the appearance of the sensor, and the communication between the various components of the device under development is also described, i.e.: the computer with the \emph{Microsoft Widnows} system, Arduino, and the module with the accelerometer and gyroscope.

\indent The next chapter is devoted to the created application, its appearance, functionalities and limitations, along with key code snippets. The script of the controller itself (Arduino), which implements the previously described designs, is also cited.

\indent The paper then presents measurements made with the sensor for various positions, from which it is possible to determine the uncertainty of the measurements. It is also shown that the developed solution does not work for dynamic tests. This fact is further analyzed.

\indent In conclusion, sample ideas for both improving and repairing the device's performance are presented.
\vspace{0.5cm}\newline
\textbf{Keywords:} calibration, accelerometer, RPY angles \vspace{0.5cm}