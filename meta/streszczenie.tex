\chapter*{Streszczenie}
\indent Praca skupia się na aspektach automatyzacji diagnostyki działania laserowego tomografu komputerowego. Wraz z opisem budowy i zasady działania LCT przedstawiono w niej fizyczne aspekty działania akcelerometrów i żyroskopów, na których ma bazować opracowywany prototyp czujnika ze sterownikiem. Opisano również zagadnienie filtru Kalmana oraz zasadność jego wykorzystania w proponowanym rozwiązaniu. Następnie przedstawiono założenie i działanie platformy Arduino oraz modułu SEN0142, a następnie opisano tryb komunikacji pomiędzy poszczególnymi elementami opracowywanego urządzenia, tj.: komputerem z systemem \emph{Microsoft Windows}, Arduino, a modułem z akcelerometrem i żyroskopem.

\indent W dalszej części pracy przedstawiona została stworzona aplikacja, jej wygląd, funkcjonalności i ograniczenia, wraz z kluczowymi fragmentami kodu. Przytoczony został również skrypt samego sterownika (Arduino), który implementuje uprzednio opisane wzory.

\indent Następnie w pracy przedstawiono pomiary wykonane czujnikiem dla różnych położeń, na podstawie których możliwe jest określenie niepewności pomiarów. Wykazano również, że opracowane rozwiązanie nie sprawdza się w przypadku badań dynamicznych. Fakt ten zostaje poddany dalszej analizie.

\indent W podsumowaniu przedstawiono przykładowe propozycje rozwiązań mające na celu zarówno ulepszenie, jak i naprawę działania urządzenia.
\vspace{0.5cm}\newline
\textbf{Słowa kluczowe:} wzorcowanie, akcelerometr, kąty RPY
\vspace{0.5cm}\newline
\noindent \textbf{Dziedzina nauki i techniki, zgodnie z wymogami OECD:} nauki inżynieryjne i techniczne, inżynieria biomedyczna
